File Descriptors
\begin{itemize}
    \item The UBLOCK contains a file descriptor table
    \begin{itemize}
        \item A file descriptor: data structure that describes the ``state" of an ``open" file. Three pieces of information
        \begin{itemize}
            \item The inode of the file
            \item The current offset in the file (location inside the file)
            \item An open flag describing what operations can be performed
        \end{itemize}
        \item File descriptor table is an array, and each file descriptor can be referenced by its index in that array. Programs treat file descriptors as this small integer
    \end{itemize}
    \item Two system calls that take a pathname and return a file descriptor
    \begin{itemize}
        \item `Creat' which creates a new file and opens it
        \item `Open' which can open an existing file (or create a new one)
    \end{itemize}
    \item Note: file suffixes don't mean anything here
\end{itemize}
Standard I/O FD's
\begin{itemize}
    \item By convention, 3 special file descriptors 0,1,2
    \begin{itemize}
        \item 0 is standard input
        \item 1 is standard output
        \item 2 is the standard error
    \end{itemize}
    \item This is just a (default) convention. No syscalls needed for this, since these file descriptors are automatically assigned (normally by programs based on this convention). Can be arranged by a parent process if the program doesn't identify a source for I/O
    \item When the parent is the shell they are either the terminal is using or redirected with $>, <$
\end{itemize}
\textbf{There are 2 system calls that take a pathname and return a file descriptor}
\begin{itemize}
    \item Creat: creates new file and opens it
    \item Open which can open an existing one (or create new one, so Open can so both)
\end{itemize}
\subsection{Creat}
\begin{itemize}
    \item int creat(const char *path\_name, mode\_t mode);
    \begin{itemize}
        \item Creates a file with the pathname
        \item Mode is the permission, for example 0644, read/write for owner, read only for group and others
    \end{itemize}
    \item If the file already exists it truncates
    \item Returns a fd, or -1 (error code)
    \begin{itemize}
        \item Sets errno on -1 to a code documented in the man page
    \end{itemize}
    \item Note: Every syscall returns -1 as error code
\end{itemize}
\textbf{Multiple ways Creat can fail}
\begin{itemize}
    \item If file found at path, tries to recycle same inode (and perms) and empty the file. However, if there are no write perms, then returns Permission denied
    \begin{itemize}
        \item File exists, don't have write permission for file
    \end{itemize}
    \item Trying to create file in root directory (which only has read and execute)
    \begin{itemize}
        \item File doesn't exist, but don't have write permission for folder (EACCESS)
    \end{itemize}
    \item Others can be found on man page
\end{itemize}

\subsection{Open (2 args)}
int open (const char *pathname, int flags);
\begin{itemize}
    \item The dual of ``creat", it just opens existing files
    \item Flags is the open flag stored in the inode
    \begin{itemize}
        \item 0 read only
        \item 1 write only
        \item 2 read and write
    \end{itemize}
\end{itemize}
\subsubsection{Open (3 args)}
int open(const char *pathname,int flags, mode\_t mode)
\begin{itemize}
    \item Can both create new files and open existing ones
    \item Mode is the same as in creat, if it creates a new file. Otherwise it just uses the existing mode
    \item Flags now have many symbolic options for bits in the flag that set corresponding bits in the Open Flag in the inode
    \begin{itemize}
        \item O\_RDONLY, O\_WRONLY, O\_RDWR are the same as 0, 1, 2
        \item O\_CREAT (create a new file if it doesn't exist)
        \item O\_TRUNC (truncate the file)
        \item We will come across many other flags like O\_NONBLOCK, O\_SYNC (wait until data gets on the disk, normally goes via buffer cache async first, which slows things since we're writing at disk speeds)
        \item Bits can be ORed together: O\_WRONLY|O\_TRUNC|O\_CREATE same as creat system call
    \end{itemize}
\end{itemize}

\subsection{Read and Write}
ssize\_t read( int fd, void *buf, size\_t count)\\
ssize\_t write(int fd, const void *buf, size\_t count)
\begin{itemize}
    \item Reads (or Writes) from (or to) the file with file descriptor fd, to (or from) the buf in the process address space
    \item Count is the size of buf, and the number of bytes requested to be read or written
    \item Return value is number of bytes actually read or written
    \begin{itemize}
        \item A return value of -1 for errno has a code
        \item Returns 0 on Read: no more bytes availble in file
    \end{itemize}
    \item Each read or write advances the location in the file by the amount read or written
    \item WARNING: the signature (void * buf) describes a pointer, which has to point to allocated storage (array of malloc'd dynamically)
    \item The OS has no knowledge of what buf is, it's just getting an address. If you give it a bad address, or an address with no enough space for the ``count" it will just try and overwrite whatever is there for count bytes
    \item Like all syscalls Reads and Writes do a \textbf{context switch} into the OS. Doing many small Reads and/or Writes are expensive since the cost of a context switch dominates the small amount of work accomplished. \textbf{Note: every syscall requires a context switch}
\end{itemize}
Real, user, sys time (time program in shell)
\begin{itemize}
    \item Real: total time
    \item Time inside program itself (just the code)
    \item Sys: how much time OS spent getting the data
\end{itemize}
Remember the buffer cache: inherently involved in disk I/O for Block devices
\begin{itemize}
    \item Note: not used for character device (DMA)
\end{itemize}
lseek
\begin{itemize}
    \item Recall that internally a file descriptor has 3 components, an inode reference, an open flag, and an offset. First two have already been discussed
    \item Offset tells the read or write syscall where in the file to operate next, its the byte count from the beginning of the file
    \item Offset is advanced as part of the read or write syscall by the amount of data read or written
    \item lseek is a syscall that alters offset in the file descriptor
    \item off\_t lseek(int fd, off\_t offset, int whence);
    \begin{itemize}
        \item whence is one of SEEK\_SET, SEEK\_CUR (offset from you are now in file), SEEK\_END (same but relative to end of file)
        \item The return value is the new offset (or -1 and errno is set)
    \end{itemize}
\end{itemize}
Misc file syscalls
\begin{itemize}
    \item pread
    \item pwrite
    \begin{itemize}
        \item Read and write using specified offset (and not changing the offset in the file descriptor)
        \item These together with lseek allow random access to file contents
    \end{itemize}
    \item readv
    \item writev
    \begin{itemize}
        \item struct iovec \{ \\void *iov\_base; /* Starting address*/\\size\_t iov\_len; /* Number of bytes to transfer */ \\\}
        \item Write/read multiple nonconsecutive intervals
        \item This allows implementation of scatter (write)/gather (read)
    \end{itemize}
\end{itemize}
\subsection{Metadata of File with Mode}
\begin{itemize}
    \item Right 9 bits are permissions  (RWX, owner, group, user)
    \begin{itemize}
        \item Affect permissions for ACCESS Prog and Database
    \end{itemize}
    \item Left 4 bits are the ``type". Single bit types are marked (ORD, DIR, CHAR, FIFO). Combinations to allow 16 types (e.g. BLOCK = CHAR | ORD)
    \item There are two userid in the UBLOCK, UID and EUID. Initially the same as login ID. EUID is used to validating permissions
    \item If SUID bit is set a process running the executable file will have its EUID take on the owner id of the file. SGID similar for groups
    \item STICKY bit
    \begin{itemize}
        \item Used differently based on type of file
        \item For ordinary files, slow to run on OS
        \item Sticky bit says, "if no nobody is using this program and there is memory, leave it in memory (don't dump it out) i.e. don't delete it". It is a suggestion/warning to OS that the program is popular and program should be left in memory if there is room
    \end{itemize}
\end{itemize}
\subsection{Stat Syscall}
\begin{itemize}
    \item int stat (const char* pathname, struct stat* statbuf);
    \item struct stat has all the metadata about a file
    \item struct timespec st\_atim; // Time of last access
    \item struct timespeect st\_mtim; // Time of last modification
    \item struct timespec st\_ctim; // Time of last status change
    \item Variants of stat syscall
    \begin{itemize}
        \item fstat (open file): for meta information on open file
        \item lstat (symbolic links): follows symbolic link and gives metadata on the file
        \item Note: cannot modify metadata with stat. We do so with chmod
    \end{itemize}
\end{itemize}
Modifying Metadata
\begin{itemize}
    \item int chmod(const char *pathname, mode\_t mode);
    \item int chown(const char *pathname, uid\_t owner, gid\_t group);
    \begin{itemize}
        \item Modify ownership permissions (file owner/group)
    \end{itemize}
    \item int utime (const char *filename, const struuct utimbuf *times);
    \begin{itemize}
        \item utimbuf is a struct with access time and modification time
        \begin{itemize}
            \item actime, modtime
        \end{itemize}
        \item Can change these times
        \item Why do this? Good for archiving programs
        \begin{itemize}
            \item For ex, zipping and unzipping files (restore time to when it was archived, NOT when it was extracted)
        \end{itemize}
        \item Utime cannot change the ctime. It can only change actime, modtime
    \end{itemize}
\end{itemize}
\section{Reading Directories}
Although Directories are a file, their internal format is not portable (lot of variations in how directories are set up in OS, e.g. character set/language, length of path name, upper/lower case, etc.). So libc/POSIX supplies a portable API
\begin{itemize}
    \item DIR *opendir(const char *name);
    \begin{itemize}
        \item This creates a directory stream, DIR is an opaque cookie efor referencing the stream
        \item Returns NULL on error and set errno
        \item It uses a struct `dirent' (directory entry), Only portable fiels are
        \begin{itemize}
            \item ino\_t d\_ino; // Inode number
            \item char d\_name[256]; // Null-terminated filename
        \end{itemize}
        \item struct dirent *readdir(DIR *dirp);
        \begin{itemize}
            \item Returns NULL at end of directory. errno distinguishes an error
        \end{itemize}
        \item int closedir(DIR *dirp);
        \begin{itemize}
            \item close dir when done
        \end{itemize}
    \end{itemize}
\end{itemize}

\section{Other Notes}
\textbf{TWO THINGS HAPPEN WHEN SYSCALL FAILS}
\begin{itemize}
    \item returns -1
    \item Changes ERRNO constant
\end{itemize}
Chmod: changes permission

use myar to create archive, ar to access it (and vice-versa?)
\begin{itemize}
    \item Flags -qU (save metadata with -U flag)
    \item ar -qU arch fileone
    \item Metadata always starts on even boundary?
    \item ar -x arch fileone
    \item -x flag restores file from archive
    \item Our file should create ar\_hdr
    \item Check /usr/include/ar.h (cat it)
    \item Each file adds ar\_hdr struct first, then file contents
    \item lseek (VERY IMPORTANT) go to end of file. lseek 0 bytes from the end
\end{itemize}
Chdir (2) changes cwd to target directory

https://sites.uclouvain.be/SystInfo/usr/include/ar.h.html

https://man7.org/linux/man-pages/man1/ar.1.html

size and mode are binary integers in struct meta???

bg command

\subsection{Synchronized vs Synchronous}
Synchronized: Syscall blocks until I/O is complete
\begin{itemize}
    \item When Write returns it means the data is on the media (e.g. Disk)
    \item UNSynchronous means the data has been copied from the user buffer but may only have gotten as far as the Buffer Cache
    \item UNSynchronized is the default
    \item Synchronous: The Syscall blocks according to the Synchronized/UNsynchronized setting
    \begin{itemize}
        \item Asynchronous means that the system call blocks only until the I/O is initiated
        \begin{itemize}
            \item On Write, the data may not have been copied from user buffer
            \item On Read, data may not have arrived yet in user buffer
        \end{itemize}
        \item Synchronous is the default
        \item SUMMARY: Default Synchronous and UNSynchronous
    \end{itemize}
\end{itemize}
Double Buffering
\begin{itemize}
    \item One approach to get more out of CPU
    \item Fill 1 buffer, fire async call for it to be processed then concurrently push to the second buffer
\end{itemize}
Changing UNSynchronized behavior
\begin{itemize}
    \item There are two basic ways
    \item Use Open Flag O\_SYNC
    \begin{itemize}
        \item Makes every write synchronized, very slow
        \item Besides specifying it at open() time, the Open Flag can also be changed while the file is open
        \begin{itemize}
            \item Eg: fcntl(fd, F\_SETFL, O\_WRONLY | O\_SYNC)
            \item Family of cntl syscalls that allow some sort of modification
        \end{itemize}
    \end{itemize}
    \item int fsync( int fd);
    \begin{itemize}
        \item Blocks until previous I/O is on the media (writing completes of file descriptor)
    \end{itemize}
    \item O\_DSYNC and fdatasync(int fd) are variations that only block on writing data, not the metadata update
    \item Asynchronous I/O requires a different paradigm
\end{itemize}
Asynchronous I/O
\begin{itemize}
    \item In this mode reading and writing are ``queued" when initiated and return before any transfer occurs
    \item Makes sense if program has something else to do while transfer occurs
    \item An AIO (async io) Control Block is used
    \item sigevent is a struct under aiocb that specifies a notification method (way for OS to signal (via SIGNAL or value) that async request has been completed)
    item opcode: if we are going to do read or write
\end{itemize}
Async Operation Syscalls
\begin{itemize}
    \item int aio\_read(struct aiocb *aiocbp);
    \item int aio\_write(struct aiocb *aiocbp);
    \item aio\_error(const struct aiocb *aiocbp);
    \begin{itemize}
        \item Return values: EINPROGRESS, ECANCELED, 0 (success)
    \end{itemize}
    \item Often used with double buffering (queue output on one buffer, and prep second, then switch buffers) and event programming (using signals)
\end{itemize}
Mandelburg skipped reading directories?
\subsection{File Descriptor Architecture}
Triple: inode number, I/O flag, offset. file descriptor ublock referred to by index
\begin{itemize}
    \item However, this isnn't fully right. File descriptors in a process UBLOCK do not really have (Inode, Flag, Offset). They are actually just pointers to a table shared by all processes, called the Open File Table
    \begin{itemize}
        \item This allows two processes to share a file descriptor
        \item That means they not only use the same file, but are updating the same offset
    \end{itemize}
    \item Open File Table itself does NOT contain the Inode, since more than one process can be using the same file with different file descriptors, independently
    \item The OS has a cache of Inodes (Inode cache and Open File Table are in BSS of OS) that are in use, ``In Memory Inodes", and the Open File Tables points here
    \item FD Tables in program U-BLOCKS (user structure) do NOT actually contain FD triples (they indirectly point to it). They are pointers to Open File Table, which actually contains the triples
    \item the I* in Open File Table is NOT actually the inode since multiple multiple I* can point to the same Inode, which is actually stored in Kernel BSS
    \item Count, in both Open File Table, and In Memory Inodes, (non-formally) indicates arrows pointing to it/in-degree
\end{itemize}
Buffered I/O
\begin{itemize}
    \item Run at buffer speeds and NOT disk speeds is huge
    \item Even though Buffer cache saves teh cost of doing small I/O reads and writes to the disk, it doesn't reduce the cost of context switches for small read and write syscalls to the Buffer Cache
    \item The `business logic' of many programs do not naturally lend themselves to doing large reads and writes to minimize the context switch overhead
\end{itemize}
More Buffered I/O
\begin{itemize}
    \item Done by using a buffered I/O library
    \item Strategy: have user API write to a library buffer in the users address space, that accumulates the small writes, and does a single large write system call when its buffer is full
    \item On reaching end of buffer, issue write syscall. This reduces number of syscalls
    \item Strategy works in reverse for reads. The library fills its buffer (in BSS) with a single large read system call, and gives it to the user API as requested. When the buffer is empty, it does another large read system call
    \item Most of the user read/write API is very fast since it is only moving data around in the address space. It runs at memory speed, not at disk speed
    \item putc, getc
\end{itemize}
Stdio Buffered I/O
\begin{itemize}
    \item Libc provides a set of stdio buffered I/O routine
    \item Uses a data structure ``FILE" which is an abstraction of a fd, for an I/O stream. Although it is opaque to the API, in actuality it contains:
    \begin{itemize}
        \item Actual ``fd" to do the system call I/O
        \item Buffer appropriately sized to the ``type" of file (ordinary, device, etc.)
        \item A ptr into the buffer for the next space for the API to get or put the next byte to satisfy the request
        \item A count of how much data (or room) is left in the buffer
        \item A flag that has info on the mode
    \end{itemize}
\end{itemize}
Special Streams
\begin{itemize}
    \item Recall the convention that programs that don't open their own files use file descriptors 0,1,2 inherited from their parent. These are called standard input, output, and error
    \begin{itemize}
        \item If the parent is the shell, $<, >, 2>$ can be used on the command line to redirect these file descriptors for the child
        \item The standard I/O library, provides three FILEs (streams) that don't need to be fopen'ed. They are associated with special file descriptors 0,1,2 and named ``stdin", ``stdout", and ``stderr"
        \begin{itemize}
            \item You have already used ``fprintf(stderr, ...). fprintf can be used with any stream, and in fact printf is really ``fprintf(stdout,..)"
        \end{itemize}
        \item \textbf{putc, and getc are equivalent to fputc and fgetc but may be macros} (faster, no overhead of getting in or out of function via stack frame)
        \item Putchar, getchar ar putc, getc to/from stdout and stdin
    \end{itemize}
\end{itemize}
FILE *fopen(const char *pathname, const char *mode);\\
FILE *fdopen(int fd, const char *mode)
\begin{itemize}
    \item Same as fopen, but uses fd instead
    \item Unlike open, the mode is string
\end{itemize}
int fputc
\begin{itemize}
    \item 
\end{itemize}
int fputs
\begin{itemize}
    \item 
\end{itemize}
int fgetc
\begin{itemize}
    \item 
\end{itemize}
char *fgets
\begin{itemize}
    \item 
\end{itemize}
size\t fread
\begin{itemize}
    \item Number of blocks of size size
\end{itemize}
size\_t fwrite
\\
Buffer cache saves cost of doing small I/O reads/writes to disk, but doesn't reduce cost of context switches for small read and write system calls to the Buffer Cache
\begin{itemize}
    \item Remember, buffered I/O has the library buf in the BSS
\end{itemize}